\documentclass{article}
\usepackage{graphicx}
\usepackage{amsmath}

\title{Introduction to Complex Numbers}
\author{mahdiar mazraei}
\date{December 29, 2023}

\begin{document}

\maketitle

\section*{Overview}

In this tutorial, we will explore the fundamentals of complex numbers. We will cover basic definitions, operations, and applications of complex numbers.\footnote{For
more information, refer to the reference books.}

\section{Definition}

A complex number is an expression of the form $a + bi$, where $a$ and $b$ are real numbers, and $i$ is the imaginary unit with the property $i^2 = -1$.

\section{Basic Operations}

The basic operations on complex numbers are summarized in Table 1.

\begin{table}[h]
\centering
\caption{Basic Operations on Complex Numbers}
\begin{tabular}{|c|c|}
\hline
Operation & Expression Result \\ \hline
Addition & $(a + bi) + (c + di) = (a + c) + (b + d)i$ \\
Subtraction & $(a + bi) - (c + di) = (a - c) + (b - d)i$ \\
Multiplication & $(a + bi) \cdot (c + di) = (ac - bd) + (ad + bc)i$ \\
Division & $\frac{a + bi}{c + di} = \frac{(a + bi)(c - di)}{c^2 + d^2}$ \\
\hline
\end{tabular}
\end{table}
\subsection{Addition and Subtraction}
To add or subtract complex numbers, simply combine the real parts and the
imaginary parts separately. For example,

\[
(a + bi) + (c + di) = (a + c) + (b + d)i.
\]

\enlargethispage{\baselineskip} % Adjust as needed

\begin{figure}
    \centering
    \includegraphics[width=0.8\linewidth]{polar.png}
    \caption{Polar Form of Complex Numbers [1].}
    \label{fig:enter-label}
\end{figure}

\subsection{Multiplication}
To multiply complex numbers, use the distributive property and remember that $i^2 = -1$. For example,
\begin{align*}
    (a + bi) \cdot (c + di) &= (ac - bd) + (ad + bc)i.
\end{align*}

\subsection{Division}
To divide complex numbers, multiply the numerator and denominator by the conjugate of the denominator. The conjugate of $c + di$ is $c - di$. For example,
\begin{align*}
    \frac{a + bi}{c + di} &= \frac{(a + bi)(c - di)}{c^2 + d^2}.
\end{align*}

\section{Polar Form}
As you see in Figure 1, complex numbers can also be represented in polar form as $r(\cos \theta + i \sin \theta)$, where $r$ is the magnitude and $\theta$ is the argument.

\section{Applications}
Complex numbers find applications in various fields, including electrical engi-
neering, physics, and signal processing. For instance, they are crucial in repre-
senting alternating currents, analyzing vibrations, and solving differential equa-
tions.
\newpage
	
\section*{Conclusion}
This tutorial provides a brief introduction to complex numbers, covering their definition, basic operations, polar form, and applications. Further exploration of this topic will unveil their significance in more advanced mathematical and scientific contexts.

\begin{thebibliography}{9}
    \bibitem{houseofmath}
    What Is the Norm and the Argument of a Complex Number?, House of Math, URL: \url{https://www.houseofmath.com}, Accessed on: December 29, 2023.
\end{thebibliography}

\end{document}
